ccc  #  Браузеры


ccc  ####  Мистраль о браузерах



#### **  Me:  **

Привет, как себя чувствуют твои нейроны?

---

#### ** Mistral:**

Привет! Мои нейроны — это всего лишь метафора, так как я — искусственный интеллект и не имею физических нейронов. Но я готов помочь тебе с любыми вопросами или задачами. Как я могу помочь?

---

--

#### **  Me:  **

-
Я использую браузер Pale Moon   что тебе известно об этом браузере и компании, его производяещей?


---
#### ** Mistral:**

Pale Moon — это веб-браузер с открытым исходным кодом, основанный на кодовой базе Mozilla Firefox. Он был создан с целью предоставить пользователям более оптимизированный и производительный браузер, сохраняя при этом совместимость с большинством расширений и тем Firefox.

### Основные особенности Pale Moon:
-  1. **Оптимизация производительности**: Pale Moon оптимизирован для максимальной производительности и эффективности использования ресурсов системы.
-  2. **Совместимость с расширениями**: Браузер поддерживает многие расширения и темы, созданные для Firefox.
-  3. **Настраиваемость**: Pale Moon предоставляет множество настроек, позволяя пользователям адаптировать браузер под свои потребности.
-  4. **Безопасность**: Браузер включает в себя различные функции безопасности, такие как блокировка вредоносных сайтов и защита от фишинга.

### Компания Moonchild Productions
Pale Moon разрабатывается и поддерживается компанией Moonchild Productions. Это небольшая независимая компания, основанная M.C. Straver (известным как Moonchild). Основная цель компании — создание и поддержка высококачественного программного обеспечения с открытым исходным кодом.

### Дополнительная информация:
- **Официальный сайт**: [Pale Moon](https://www.palemoon.org/)
- **Форум и сообщество**: Pale Moon имеет активное сообщество пользователей и разработчиков, где можно найти поддержку и обсудить различные аспекты использования браузера.

Если у тебя есть конкретные вопросы о Pale Moon или его использовании, не стесняйся спрашивать!


---

#### **  Me:  **

да, есть. Разъясни суть следующих апдейтов:
-  Changed the way cookies are handled internally to fix an issue with cookie database corruption as a result of updates to domain suffixes.
-  Fixed an issue with Alternative-Services protocol negotiation.
-  Fixed a potential crash scenario with Structured Clone operations. DiD
-  Fixed a potential issue with line breaking if out of memory.
-  Fixed a rare crash with opportunistic encryption.
-  Minor code cleanup.
-  Security issues addressed: CVE-2025-0239 and CVE-2025-0238.

---

#### ** Mistral:**

Конечно, давай разберем каждый из этих апдейтов:

-  1. **Changed the way cookies are handled internally to fix an issue with cookie database corruption as a result of updates to domain suffixes.**
   - *Суть*: Изменено внутреннее обращение с куки (cookies) для исправления проблемы с повреждением базы данных куки, вызванной обновлениями доменных суффиксов.
   - *Объяснение*: Доменные суффиксы (например, .com, .org) могут изменяться, и это может приводить к повреждению базы данных куки. Обновление исправляет эту проблему, изменяя способ хранения и обработки куки.

-  2. **Fixed an issue with Alternative-Services protocol negotiation.**
   - *Суть*: Исправлена проблема с переговорами по протоколу Alternative-Services.
   - *Объяснение*: Alternative-Services — это протокол, который позволяет серверам предлагать альтернативные сервисы для клиентов. Обновление исправляет ошибки, связанные с этим процессом.

-  3. **Fixed a potential crash scenario with Structured Clone operations. DiD**
   - *Суть*: Исправлен потенциальный сценарий сбоя при операциях Structured Clone.
   - *Объяснение*: Structured Clone — это механизм для копирования сложных объектов в JavaScript. Обновление исправляет возможные сбои, связанные с этими операциями.

-  4. **Fixed a potential issue with line breaking if out of memory.**
   - *Суть*: Исправлена потенциальная проблема с переносом строк при нехватке памяти.
   - *Объяснение*: При нехватке оперативной памяти могут возникать проблемы с переносом строк в тексте. Обновление исправляет эту проблему, обеспечивая корректное отображение текста даже при ограниченных ресурсах.

-  5. **Fixed a rare crash with opportunistic encryption.**
   - *Суть*: Исправлен редкий сбой, связанный с оппортунистическим шифрованием.
   - *Объяснение*: Оппортунистическое шифрование — это метод шифрования данных, когда это возможно, без необходимости предварительной настройки. Обновление исправляет редкие сбои, связанные с этим методом.

-  6. **Minor code cleanup.**
   - *Суть*: Мелкая очистка кода.
   - *Объяснение*: Внесены небольшие изменения в код для улучшения его читаемости и поддерживаемости. Это может включать удаление устаревшего кода, исправление комментариев и другие мелкие улучшения.

-  7. **Security issues addressed: CVE-2025-0239 and CVE-2025-0238.**
   - *Суть*: Устранены уязвимости безопасности CVE-2025-0239 и CVE-2025-0238.
   - *Объяснение*: CVE (Common Vulnerabilities and Exposures) — это система каталогизации известных уязвимостей. Обновление исправляет конкретные уязвимости, указанные под номерами CVE-2025-0239 и CVE-2025-0238, улучшая общую безопасность браузера.

--  Эти обновления направлены на улучшение стабильности, безопасности и производительности браузера Pale Moon.

---

#### **  Me:  **

- 
Какие еще альтернативные Оупенсорс браузеры ты бы порекомендовал использовать?


---



#### ** Mistral:**
* 
Конечно! Вот несколько альтернативных браузеров с открытым исходным кодом, которые ты можешь рассмотреть:

-  1. **Mozilla Firefox**
   - *Описание*: Один из самых популярных браузеров с открытым исходным кодом. Известен своей гибкостью, множеством расширений и акцентом на конфиденциальность.
   - *Особенности*: Поддержка множества расширений, регулярные обновления безопасности, встроенные инструменты для защиты конфиденциальности.

-  2. **Brave**
   - *Описание*: Браузер, ориентированный на конфиденциальность и безопасность. Основывается на Chromium, но включает в себя дополнительные функции для блокировки рекламы и отслеживания.
   - *Особенности*: Встроенная блокировка рекламы, защита от отслеживания, поддержка криптовалют.

-  3. **Chromium**
   - *Описание*: Открытая версия браузера Google Chrome. Предоставляет ту же производительность и совместимость с расширениями, что и Chrome, но без некоторых проприетарных компонентов.
   - *Особенности*: Высокая производительность, поддержка множества расширений, регулярные обновления.

-  4. **Tor Browser**
   - *Описание*: Браузер, ориентированный на анонимность и конфиденциальность. Использует сеть Tor для маршрутизации трафика через несколько узлов, скрывая твое местоположение.
   - *Особенности*: Анонимность, защита от отслеживания, встроенные инструменты для защиты конфиденциальности.

-  5. **Waterfox**
   - *Описание*: Браузер, основанный на Firefox, но с акцентом на конфиденциальность и производительность. Поддерживает старые расширения Firefox, которые могут не работать в последних версиях.
   - *Особенности*: Поддержка старых расширений, акцент на конфиденциальность, оптимизация производительности.

-  6. **LibreWolf**
   - *Описание*: Форк Firefox, ориентированный на конфиденциальность и безопасность. Удаляет все телеметрические и проприетарные компоненты.
   - *Особенности*: Удаление телеметрии, акцент на конфиденциальность, регулярные обновления.

-  7. **Midori**
   - *Описание*: Легкий и быстрый браузер, разработанный с использованием GTK. Подходит для пользователей, ищущих простой и минималистичный браузер.
   - *Особенности*: Легкость, быстрота, простота использования.

-  8. **Qutebrowser**
   - *Описание*: Минималистичный браузер, управляемый с помощью клавиатуры. Ориентирован на пользователей, предпочитающих командную строку и минимализм.
   - *Особенности*: Управление с клавиатуры, минималистичный интерфейс, высокая производительность.

-- Каждый из этих браузеров имеет свои уникальные особенности и преимущества, так что выбор зависит от твоих конкретных потребностей и предпочтений.