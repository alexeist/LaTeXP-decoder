ccc # **Команды форматирования LaTexP**

-- ccc  ###  **Расширенные команды (только в LaTexP)**
- 
   ######      "######" в начале строки ->   tag h6.
   #####       "#####"  в начале строки ->   tag h5. 
   ####        "####"   в начале строки ->   tag h4.

   ###         "###" в начале строки -> tag h3.
   ##          "##" в начале строки -> tag h2.
   #           "#" в начале строки -> tag h1.

 
###  ccc  "ccc "  в начале строки форматировать по центру
***Выражение помещенное в тройные звездочки***
* 
  -выделяет красным цветом
-  
**Выражение помещенное в двойные звездочки**  -выделяет синим цветом
-  * Выражение помещенное в одинарные звездочки*  -выделяет болдом (сделано для совместимости с текстами из Вайбера и других соцсетей)
* 
-     -__        (минус плюс 2 пробела) вставляет перенос строки (дефисы (-) и математические минусы \(-\) ставьте без пробелов после них)

--     --_       (два минуса плюс пробел)вставляет перенос строки и строку выделяет курсивом

-     "-""-""-"  (три тире подряд) вставляет разделительную линию

-   ```  ` ` `  (три обратные кавычки) делает строку жирным шрифтом

-  //           (двойной слеш и пробел) - комментарый ббудет выведен серым цветом

-   ! ! !       (три воскл. знака подряд) -  текст вообще не выводится

-  !!!          это невидимый комментарий он не выводится на экран его можно использовать для установки специальных слов, напр титлов браузера, адреса картинок и под. 

-  ___          (_ _ _)  -тройное подчеркивание делает строку уменьшенным шрифтом
---  

-   
-  
-   
ccc  ###  **Основные команды LaTex**
-- 
- Выражения, подлежащее обработке MathJax надо помещать в экранированные обратным слешем скобки:

-    круглые =>  формула будет в строке, 
-    квадратные=>  формула будет отформатирована по центру 
-  
 

--   ###### ПРИМЕР:


 \ ( содержание формулы  \ )  для форматирования в строке

- \ [ содержание формулы  \ ] для форматирования по центру страницы  
-  
---
- 
-  
ccc ##  Кодировка MathJax


--   **1. Математические символы:**
   - `+`, `-`, `*`, `/`, `=`
 \[
     {+} ,  {-} ,  {* },  {/} ,  {=} 
 \]
   - `\alpha`, `\beta`, `\gamma`, `\delta`, `\epsilon`, `\zeta`, `\eta`, `\theta`, `\iota`, `\kappa`, `\lambda`, `\mu`, `\nu`, `\xi`, `\pi`, `\rho`, `\sigma`, `\tau`, `\upsilon`, `\phi`, `\chi`, `\psi`, `\omega`
 \[
     {\alpha}, {\beta}, {\gamma}, {\delta}, {\epsilon}, {\zeta}, {\eta}, {\theta}, {\iota}, {\kappa}, {\lambda}, {\mu}, {\nu}, {\xi}, {\pi}, {\rho}, {\sigma}, {\tau}, {\upsilon}, {\phi}, {\chi}, {\psi}, {\omega}
 \]
   - `\Gamma`, `\Delta`, `\Theta`, `\Lambda`, `\Xi`, `\Pi`, `\Sigma`, `\Upsilon`, `\Phi`, `\Psi`, `\Omega`
\[   
      {\Gamma}, {\Delta}, {\Theta}, {\Lambda}, {\Xi}, {\Pi}, {\Sigma}, {\Upsilon}, {\Phi}, {\Psi}, {\Omega}
\]
---
--  **2. Функции**:
 ccc  - `\sin`, `\cos`, `\tan`, `\cot`, `\sec`, `\csc`
 ccc   - `\arcsin`, `\arccos`, `\arctan`
 ccc  - `\sinh`, `\cosh`, `\tanh`, `\coth`
 ccc  - `\log`, `\ln`, `\exp`
 ccc  - `\max`, `\min`
\[ {\sin}, {\cos}, {\tan}, {\cot}, {\sec}, {\csc},\]
\[   {\arcsin}, {\arccos}, {\arctan},\]
\[   {\sinh}, {\cosh}, {\tanh}, {\coth},\]
\[    {\log}, {\ln}, {\exp},\]
\[   {\max}, {\min}\]
   
---

-- **3. Специальные символы**:
ccc     `\infty`, `\partial`, `\nabla`, `\forall`, `\exists`, `\emptyset`, `\varnothing`
ccc     `\mathbb{R}`, `\mathbb{C}`, `\mathbb{Q}`, `\mathbb{Z}`, `\mathbb{N}`
ccc    `\mathcal{A}`, `\mathcal{B}`, `\mathcal{C}`, `\mathcal{D}`, `\mathcal{E}`, `\mathcal{F}`, `\mathcal{G}`, 
ccc    `\mathcal{H}`, `\mathcal{I}`, `\mathcal{J}`, `\mathcal{K}`, `\mathcal{L}`, `\mathcal{M}`, `\mathcal{N}`, 
ccc    `\mathcal{O}`, `\mathcal{P}`, `\mathcal{Q}`, `\mathcal{R}`, `\mathcal{S}`, `\mathcal{T}`, `\mathcal{U}`, 
ccc    `\mathcal{V}`, `\mathcal{W}`, `\mathcal{X}`, `\mathcal{Y}`, `\mathcal{Z}`

\[{\infty}, {\partial}, {\nabla}, {\forall}, {\exists}, {\emptyset}, {\varnothing},\]
\[{\mathbb{R}}, {\mathbb{C}}, {\mathbb{Q}}, {\mathbb{Z}}, {\mathbb{N}},\]
\[{\mathcal{A}}, {\mathcal{B}}, {\mathcal{C}}, {\mathcal{D}}, {\mathcal{E}}, {\mathcal{F}}, {\mathcal{G}}, \]
\[{\mathcal{H}}, {\mathcal{I}}, {\mathcal{J}}, {\mathcal{K}}, {\mathcal{L}}, {\mathcal{M}}, {\mathcal{N}}, \]
\[{\mathcal{O}}, {\mathcal{P}}, {\mathcal{Q}}, {\mathcal{R}}, {\mathcal{S}}, {\mathcal{T}}, {\mathcal{U}}, \]
\[{\mathcal{V}}, {\mathcal{W}}, {\mathcal{X}}, {\mathcal{Y}}, {\mathcal{Z}},\]

---  
 

--  **4. Дробные и корневые выражения**:
ccc   - `\frac{a}{b}`,  `\sqrt{x}`, `\sqrt[n]{x^2}`
\[ \frac{a}{b},  \sqrt{x}, \sqrt[n]{x^2}\]

---   


--  **5. Суммы и интегралы**:
ccc   - `\sum`, `\prod`, `\int`, `\oint`, `\iint`, `\iiint`, `\iiiint`
 \[ 
   {\sum}, {\prod}, {\int}, {\oint}, {\iint}, {\iiint}, {\iiiint}
 \]
----
 
--  **6. Матрицы**:
--       
 
ccc -  '\ begin{matrix} a & b \\ c & d \ end{matrix}'
\[
   \begin{matrix} a & b \\ c & d \end{matrix}
\]
   
ccc -  `\ begin{pmatrix} a & b \\ c & d \end{pmatrix}` 
  \[ \begin{pmatrix} a & b \\ c & d \end{pmatrix}\]
   
ccc - `\ begin{bmatrix} a & b \\ c & d \end{bmatrix}`
    \[\begin{bmatrix} a & b \\ c & d \end{bmatrix}\]
   
ccc -    `\ begin{vmatrix} a & b \\ c & d \end{vmatrix}`
    \[\begin{vmatrix} a & b \\ c & d \end{vmatrix}\]

ccc -  `\ begin{Vmatrix} a & b \\ c & d \end{Vmatrix}`
     \[\begin{Vmatrix} a & b \\ c & d \end{Vmatrix}\]
---

--  **7. Операторы**:

ccc -   `\pm`, `\mp`, `\times`, `\div`, `\cdot`, `\ast`, `\star`, `\circ`, `\bullet`, `\oplus`,
ccc -   `\ominus`, `\otimes`, `\oslash`, `\odot`, `\dagger`, `\ddagger`, `\cap`, `\cup`, `\uplus`, 
ccc -   `\sqcap`, `\sqcup`, `\triangleleft`, `\triangleright`, `\bigtriangleup`, `\bigtriangledown`, 
ccc -   `\lhd`, `\rhd`, `\unlhd`, `\unrhd`
\[        \pm,     \mp, \times,   \div, \cdot,   \ast,    \star,    \circ, \bullet, \oplus, \]
\[        \ominus, \otimes, \oslash, \odot, \dagger, \ddagger, \cap, \cup, \uplus, \]
\[        \sqcap, \sqcup, \triangleleft, \triangleright, \bigtriangleup, \bigtriangledown, \]
\[        \lhd, \rhd, \unlhd, \unrhd\]

---


-- **8. Стрелки**:
ccc -    `\leftarrow`, `\rightarrow`, `\uparrow`, `\downarrow`
ccc -    `\Leftarrow`, `\Rightarrow`, `\Uparrow`, `\Downarrow`
ccc -    `\longleftarrow`, `\longrightarrow`, `\Longleftarrow`, `\Longrightarrow`
ccc -    `\mapsto`,
ccc -    `\hookleftarrow`, `\hookrightarrow`, `\leftharpoonup`, `\rightharpoonup`,
ccc -    `\leftharpoondown`, `\rightharpoondown`, `\rightleftharpoons`, `\leftrightarrows`


\[       \leftarrow ,  \rightarrow ,  \uparrow ,  \downarrow , \]
\[       \Leftarrow ,  \Rightarrow ,  \Uparrow ,  \Downarrow ,\]
\[       \longleftarrow ,  \longrightarrow ,  \Longleftarrow ,  \Longrightarrow , \] 
\[       \mapsto , \]
\[       \hookleftarrow ,  \hookrightarrow ,  \leftharpoonup ,  \rightharpoonup ,\]
\[      \leftharpoondown ,  \rightharpoondown ,  \rightleftharpoons ,  \leftrightarrows \] 


--- 
--  **9. Разделители**:
ccc   - `\left(`, `\right)`, `\left[`, `\right]`, `\left\{`, `\right\}`, `\left|`, `\right|`, `\left\|`, `\right\|`
   \[
   
     \left( ,  \right) ,  \left[ ,  \right] ,  \left\{ ,  \right\} ,  \left| ,  \right| ,  \left\| ,  \right\| 
   \]
--- 

--  **10. Прочие**:
ccc    - `\ldots`, `\cdots`, `\vdots`, `\ddots`
    - `\hat{a}`, `\check{a}`, `\breve{a}`, `\acute{a}`, `\grave{a}`, `\tilde{a}`, `\bar{a}`, `\vec{a}`, `\dot{a}`, `\ddot{a}`
\[
     \ldots ,  \cdots ,  \vdots ,  \ddots, 
      \hat{a} ,  \check{a} ,  \breve{a} ,  \acute{a} ,  \grave{a} ,  \tilde{a} ,  \bar{a} ,  \vec{a} ,  \dot{a} ,  \ddot{a} 


\]
---
-- 
-- 
ccc  ### Примеры использования
-- 
 **1.Простая дробь**:
   ```latex
   \frac{a}{b}
   ```
\[    \frac{a}{b} \]

-- **2. Квадратный корень**:
   ```latex
   \sqrt{x}
   ```
\[   \sqrt{x} \]
-- **3. Сумма**:
   ```latex
   \sum_{i=1}^{n} i
   ```
   \[ \sum_{i=1}^{n} i \]

--  **4. Интеграл**:
   ```latex
   \int_{a}^{b} f(x) \, dx
   ```
   \[ \int_{a}^{b} f(x) \, dx \]

--  **5. Матрица**:
   ```latex
   \ begin{pmatrix}
   a & b \\ 
   c & d 
   \ end{pmatrix} 
 ```

\[
\begin{pmatrix}
   a & b \\
   c & d
   \end{pmatrix}


\]

---
--  

  

ccc ## Расшифровка некоторых математических символов

-- ### **1. Гиперболический синус (sinh)**

Гиперболический синус — это гиперболическая функция, аналогичная тригонометрическому синусу. Она определяется следующим образом:

\[ \sinh(x) = \frac{e^x {-} e^{{-}x}}{2} \]

-- ##### Где:
 \( e \) — основание натурального логарифма (примерно равно 2.71828).
- \( x \) — аргумент функции.
-- 

-- 
-- 
--  ### **2. Гиперболический косинус (cosh)**

Гиперболический косинус — это гиперболическая функция, аналогичная тригонометрическому косинусу. Она определяется следующим образом:

\[ \cosh(x) = \frac{e^x + e^{{-}x}}{2} \]

-- #####  Где:
 \( e \) — основание натурального логарифма.
- \( x \) — аргумент функции.
-- 
-- 
--

--  ### **3. Произведение**  `(\prod)`

Символ произведения \(\prod\) используется для обозначения произведения последовательности чисел. Он аналогичен символу суммы \(\sum\), но вместо сложения элементов последовательности, он их перемножает.

 


\[ \prod_{i=1}^{n} a_i = a_1 \cdot a_2 \cdot \ldots \cdot a_n \]

-- #####  Где:
\( i \) — индекс элемента последовательности.
- \( a_i \) — элементы последовательности.
- \( n \) — количество элементов в последовательности.

---

--  ### **4. Круговой интеграл**
 -`(\oint)`

Символ кругового интеграла \(\oint\) используется для обозначения интеграла по замкнутому контуру. Он часто встречается в комплексном анализе и векторном исчислении.
--
-- ##### Пример использования  кругового интеграла:

\[ \oint_C f(z) \, dz \]

-- ##### Где:
\( C \) — замкнутый контур.
- \( f(z) \) — функция, интегрируемая по контуру.
- \( z \) — комплексная переменная.



